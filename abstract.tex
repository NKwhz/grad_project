% -*- coding: utf-8 -*-


\begin{zhaiyao}
韵律文本是一种特殊类型的文本,相对于通俗的自然语言,韵律文本分析与学习值得进行专门的研究。常见的韵律文本主要包括诗歌、歌词和一些其他艺术形式中的唱词。与一般的自然语言不同,韵律文本通常是有一定音韵特征的。韵律文本的韵律是其主要特征,在进行韵律文本学习时,除了其语义特征,韵律特征也是需要着重关注的一点。\par

现有韵律学习方法大多需要人为规定一些韵律特征来进行学习,甚至完全忽略韵律文本的韵律特征,仅仅考虑其语义特征及其形式。\par

为了解决以上问题,本文提出了一个新方法,即rhyme2vec,来学习韵律表征向量。这个方法包含两个模型,即连续行韵律和隔行韵律。通过整合这两个模型,rhyme2vec可以很好地处理韵律模式的多种特征。\par

本文还提出了一个结合了层次注意力机制的变分自编码器的框架用于融合韵律文本的韵律特征和语义特征。该框架旨在处理韵律文本的表征学习问题,整合了多种未被探索的机制,即,利用注意力机制对韵律信息进行有效整合以及对语义与韵律信息进行无缝整合。\par

最终,本文通过实验验证了rhyme2vec和层次注意力机制变分自编码器框架训练得到的表征向量在检索和分类等任务上的有效性,实验包括下一行预测、流派分类、歌词生成。通过与现有的一些表现较好的韵律文本学习方法进行比较,最终结果显示rhyme2vec和层次注意力机制变分自编码器框架相对于这些方法更为有效。


\end{zhaiyao}




\begin{guanjianci}
韵律文本学习;注意力机制;变分自编码器;表示学习
\end{guanjianci}



\begin{abstract}
Rhyme text is a special type of text. Compared with common natural language, prosodic text analysis and learning are worthy of specified study. Common rhyme texts mainly include poetry, lyrics, and lyrics in some other art forms. Unlike common natural language, rhyme texts usually have certain prosodic features. The rhymes of rhyme texts are their main feature. In addition to their semantic features, rhyme features need to be paid attention to.\par

Most of existing rhyme learning methods require manually provided rhyme features to learn, or even completely ignore the prosodic features of rhyme texts, only considering their semantic features and forms.\par

In order to solve the above problems, this paper proposes a new method, namely rhyme2vec, to learn the prosodic representation vector. This method consists of two models, continuous line rhyme and interlaced rhyme. By integrating these two models, rhyme2vec can handle many features of prosodic patterns well.\par

This paper also proposes a framework of variational autoencoders combined with hierarchical attention mechanisms to fuse prosodic features and semantic features of rhyme texts. The framework is designed to deal with the problem of representation learning of rhyme texts, incorporating a variety of unexplored mechanisms, namely, the use of attentional mechanisms for the effective integration of prosodic information and the seamless integration of semantic and prosodic information.\par

In the end, the experiment verifies the effectiveness of the representation vectors obtained by rhyme2vec and the hierarchical attentional mechanism based variational autoencoder framework on tasks such as retrieval and classification. The experiments include the next-line prediction, genre classification, and lyrics generation. By comparing with some state-of-the-art prosodic text learning methods, the final results show that rhyme2vec and the hierarchical attentional mechanism based variational autoencoder frameworks outperform these methods.\par

\end{abstract}



\begin{keywords}
Rhyme Texts Learning; Attention Mechanism; Variational Autoencoder; Representation Learning
\end{keywords} 